%*******************************************************************************
% Definicje stylu dokumentu
%*******************************************************************************

%===============================================================================
% klasa dokumentu

%\documentclass[12pt, a4paper, twoside, titlepage, final]{mwbk}
\documentclass[10pt,a4paper,onecolumn,twoside,11pt,wide,floatssmall]{book}
%\documentclass[10pt,a4paper,onecolumn,oneside,11pt,wide,floatssmall]{article}


%===============================================================================
% Pakiety
%\usepackage[latin2]{inputenc}
\usepackage{polski}
%\usepackage[cp1250]{inputenc}
\usepackage[utf8]{inputenc}				% kodowanie �r�d�a
\usepackage[polish]{babel}				% polskie przenoszenie wyraz�w (hyph.)
\usepackage[T1]{fontenc}					% font PL
\usepackage{url}								% polecenie \url
\usepackage{amsfonts}						% fonty matematyczne
\usepackage{graphicx}						% wstawianie grafiki
\usepackage{color}							% kolory
\usepackage{fancyhdr}						% paginy g�rne i dolne
\usepackage[plainpages=false]{hyperref}		% dynamiczne linki
\usepackage{calc}							% operacje arytmetyczne w TeX'u
\usepackage{tabularx}						% rozci�gliwe tabele
\usepackage{array}							% standardowe tabele
\usepackage{geometry}
\usepackage{hyperref}
\usepackage{subfigure}
\usepackage{wrapfig}
\usepackage{indentfirst}
\usepackage{amsmath}
\usepackage{color}
\usepackage{array}
\usepackage{pdflscape}
\usepackage{amsmath}
\usepackage{textcomp}
\usepackage[font={small,it}]{caption}
\usepackage{etoolbox}
\usepackage[section]{placeins}
\usepackage{float}
\graphicspath{{./img/}}
\usepackage{listings}             % Include the listings-package
\usepackage{enumitem}


\linespread{1.3}								% 1.3 do interlinii 1.5


\patchcmd{\thebibliography}{\chapter*}{\section*}{}{}

\bibliographystyle{plain}
% w�asne pakiety

%===============================================================================
% Ustawienia dokumentu

\frenchspacing

% ustawienia wymiar�w
\oddsidemargin 0mm							% margines nieparzystych stron
\evensidemargin 0mm							% margines parzystych stron
\headheight 15pt								% wysoko�� paginy g�rnej
\topmargin 0mm									% margines g�rny
\setlength{\parindent}{0pt}
\setlength{\parskip}{1ex plus 0.5ex minus 0.2ex}
% styl paginacji
\pagestyle{fancy}
% \renewcommand{\chaptermark}[1]{}%{\markboth{#1}{}} % BO ARTICLE
\renewcommand{\sectionmark}[1]{}%{\markright{\thesection\ #1}{}}
\renewcommand{\thesection}{\arabic{section}}


% nag��wek 
\fancyhf{}

\fancyhead[RE,LO]{\thepage}
\fancyhead[LE,RO]{[GIS]~Teoria grafów w modelowaniu epidemii}
\renewcommand{\headrulewidth}{0.1pt}
\renewcommand{\footrulewidth}{0pt}

% nag��wek w stylu plain 
\fancypagestyle{plain}
{
\fancyhf{}
\renewcommand{\headrulewidth}{0pt}
\renewcommand{\footrulewidth}{0pt}
}

% ta sekwencja tworzy czyste kartki na stronach po \cleardoublepage
\makeatletter
\def\cleardoublepage{\clearpage\if@twoside \ifodd\c@page\else
	\hbox{}
	\vspace*{\fill}
	\thispagestyle{empty}
	\newpage
	\if@twocolumn\hbox{}\newpage\fi\fi\fi}
\makeatother

%===============================================================================
% Zmienne �rodowiskowe i polecenia

% definicja
% \newtheorem{definition}{Definicja}[chapter] % BO CHAPTER
\newtheorem{definition}{Definicja}

% twierdzenie
% \newtheorem{theorem}{Twierdzenie}[chapter] % BO CHAPTER
\newtheorem{theorem}{Twierdzenie}

% obcoj�zyczne nazwy
\newcommand{\foreign}[1]{\emph{#1}}

% pozioma linia
\newcommand{\horline}{\noindent\rule{\textwidth}{0.4mm}}

% wstawianie obrazk�w {plik}{caption}{opis}
\newcommand{\fig}[3]
{
\begin{figure}[!htb]
\begin{center}
\includegraphics[width=\textwidth]{#1}
\caption[#2]{#2. #3}
\label{#1}
\end{center}
\end{figure}
}

%===============================================================================
% ustawienia pakietu hyperref

\hypersetup
{
%colorlinks=true,			% false: boxed links; true: colored links
%linkcolor=black,			% color of internal links
%citecolor=black,			% color of links to bibliography
%filecolor=black,			% color of file links
%urlcolor=black			% color of external links
}

%===============================================================================

% Custom titlepage
\makeatletter
\newcommand*{\subtitle}[1]{\newcommand*{\@subtitle}{#1}}
\newcommand*{\supervisor}[1]{\newcommand*{\@supervisor}{#1}}
\newcommand*{\course}[1]{\newcommand*{\@course}{#1}}
\newcommand*{\coursecode}[1]{\newcommand{\@coursecode}{#1}}
\newcommand*{\university}[1]{\newcommand{\@university}{#1}}
\newcommand*{\faculty}[1]{\newcommand{\@faculty}{#1}}
\newcommand{\sepline}{\rule{\linewidth}{0.4mm}}
% ------------------------------------------------------------------------------ 
\renewcommand*{\maketitle}{%
	\@ifundefined{@course}{%
		\providecommand*{\@course}{\VariableNotSetFix{\course}}%
		
	}{}
	

	\hypersetup{pageanchor=false}
	\begin{titlepage}%
		\vspace*{1cm}
		
		\begin{figure}[H]
			\centering
			\includegraphics[width=3cm]{weiti_logo.png} % Include a department/university logo - this will require the graphicx package
			\label{fig:logo}
		\end{figure}
		\begin{center}%
			\@ifundefined{@university}{}{%
				{\small \@university \par}}%
			\@ifundefined{@faculty}{}{%
				{\small \@faculty \par}}%
		\end{center}
		%\vspace{1cm}
		\begin{center}%
			\@ifundefined{@coursecode}{%
				{\Large \@course \par}%
			}{%
			{\Large \@course \ (\@coursecode) \par}}%
	\end{center}
	\sepline
	\begin{center}%
		{\LARGE \textsc{\@title} \par}
		\@ifundefined{@subtitle}{}%
		{\vspace{0.5cm}%
			\normalsize \textsc{\@subtitle} \par}%
	\end{center}
	\sepline
	\vspace{1cm}
	\begin{center}%
		{\normalsize
			\itshape
			\lineskip .5em%
			\begin{tabular}[t]{c}%
				\@author
			\end{tabular}\par}%
	\end{center}
	\vspace{2cm}
	\@ifundefined{@supervisor}{}{%	
		\begin{flushright}%
			{\normalsize Opiekun projektu: \par}
			{\normalsize \itshape \@supervisor \par}
		\end{flushright}		
		
	}
	%\vfill\null
	\begin{center}%
		\@date
	\end{center}
	
	\@thanks
\end{titlepage}%
\hypersetup{pageanchor=true}
\setcounter{footnote}{0}%
\global\let\thanks\relax
\global\let\maketitle\relax
\global\let\@thanks\@empty
\global\let\@author\@empty
\global\let\@date\@empty
\global\let\@title\@empty
\global\let\title\relax
\global\let\author\relax
\global\let\date\relax
\global\let\and\relax

}

\makeatother