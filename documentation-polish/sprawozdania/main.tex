%*******************************************************************************
% Definicje stylu dokumentu
%*******************************************************************************

%===============================================================================
% klasa dokumentu

%\documentclass[12pt, a4paper, twoside, titlepage, final]{mwbk}
\documentclass[10pt,a4paper,onecolumn,twoside,11pt,wide,floatssmall]{book}
%\documentclass[10pt,a4paper,onecolumn,oneside,11pt,wide,floatssmall]{article}


%===============================================================================
% Pakiety
%\usepackage[latin2]{inputenc}
\usepackage{polski}
%\usepackage[cp1250]{inputenc}
\usepackage[utf8]{inputenc}				% kodowanie �r�d�a
\usepackage[polish]{babel}				% polskie przenoszenie wyraz�w (hyph.)
\usepackage[T1]{fontenc}					% font PL
\usepackage{url}								% polecenie \url
\usepackage{amsfonts}						% fonty matematyczne
\usepackage{graphicx}						% wstawianie grafiki
\usepackage{color}							% kolory
\usepackage{fancyhdr}						% paginy g�rne i dolne
\usepackage[plainpages=false]{hyperref}		% dynamiczne linki
\usepackage{calc}							% operacje arytmetyczne w TeX'u
\usepackage{tabularx}						% rozci�gliwe tabele
\usepackage{array}							% standardowe tabele
\usepackage{geometry}
\usepackage{hyperref}
\usepackage{subfigure}
\usepackage{wrapfig}
\usepackage{indentfirst}
\usepackage{amsmath}
\usepackage{color}
\usepackage{array}
\usepackage{pdflscape}
\usepackage{amsmath}
\usepackage{textcomp}
\usepackage[font={small,it}]{caption}
\usepackage{etoolbox}
\usepackage[section]{placeins}
\usepackage{float}
\graphicspath{{./img/}}
\usepackage{listings}             % Include the listings-package
\usepackage{enumitem}


\linespread{1.3}								% 1.3 do interlinii 1.5


\patchcmd{\thebibliography}{\chapter*}{\section*}{}{}

\bibliographystyle{plain}
% w�asne pakiety

%===============================================================================
% Ustawienia dokumentu

\frenchspacing

% ustawienia wymiar�w
\oddsidemargin 0mm							% margines nieparzystych stron
\evensidemargin 0mm							% margines parzystych stron
\headheight 15pt								% wysoko�� paginy g�rnej
\topmargin 0mm									% margines g�rny
\setlength{\parindent}{0pt}
\setlength{\parskip}{1ex plus 0.5ex minus 0.2ex}
% styl paginacji
\pagestyle{fancy}
% \renewcommand{\chaptermark}[1]{}%{\markboth{#1}{}} % BO ARTICLE
\renewcommand{\sectionmark}[1]{}%{\markright{\thesection\ #1}{}}
\renewcommand{\thesection}{\arabic{section}}


% nag��wek 
\fancyhf{}

\fancyhead[RE,LO]{\thepage}
\fancyhead[LE,RO]{[GIS]~Teoria grafów w modelowaniu epidemii}
\renewcommand{\headrulewidth}{0.1pt}
\renewcommand{\footrulewidth}{0pt}

% nag��wek w stylu plain 
\fancypagestyle{plain}
{
\fancyhf{}
\renewcommand{\headrulewidth}{0pt}
\renewcommand{\footrulewidth}{0pt}
}

% ta sekwencja tworzy czyste kartki na stronach po \cleardoublepage
\makeatletter
\def\cleardoublepage{\clearpage\if@twoside \ifodd\c@page\else
	\hbox{}
	\vspace*{\fill}
	\thispagestyle{empty}
	\newpage
	\if@twocolumn\hbox{}\newpage\fi\fi\fi}
\makeatother

%===============================================================================
% Zmienne �rodowiskowe i polecenia

% definicja
% \newtheorem{definition}{Definicja}[chapter] % BO CHAPTER
\newtheorem{definition}{Definicja}

% twierdzenie
% \newtheorem{theorem}{Twierdzenie}[chapter] % BO CHAPTER
\newtheorem{theorem}{Twierdzenie}

% obcoj�zyczne nazwy
\newcommand{\foreign}[1]{\emph{#1}}

% pozioma linia
\newcommand{\horline}{\noindent\rule{\textwidth}{0.4mm}}

% wstawianie obrazk�w {plik}{caption}{opis}
\newcommand{\fig}[3]
{
\begin{figure}[!htb]
\begin{center}
\includegraphics[width=\textwidth]{#1}
\caption[#2]{#2. #3}
\label{#1}
\end{center}
\end{figure}
}

%===============================================================================
% ustawienia pakietu hyperref

\hypersetup
{
%colorlinks=true,			% false: boxed links; true: colored links
%linkcolor=black,			% color of internal links
%citecolor=black,			% color of links to bibliography
%filecolor=black,			% color of file links
%urlcolor=black			% color of external links
}

%===============================================================================

% Custom titlepage
\makeatletter
\newcommand*{\subtitle}[1]{\newcommand*{\@subtitle}{#1}}
\newcommand*{\supervisor}[1]{\newcommand*{\@supervisor}{#1}}
\newcommand*{\course}[1]{\newcommand*{\@course}{#1}}
\newcommand*{\coursecode}[1]{\newcommand{\@coursecode}{#1}}
\newcommand*{\university}[1]{\newcommand{\@university}{#1}}
\newcommand*{\faculty}[1]{\newcommand{\@faculty}{#1}}
\newcommand{\sepline}{\rule{\linewidth}{0.4mm}}
% ------------------------------------------------------------------------------ 
\renewcommand*{\maketitle}{%
	\@ifundefined{@course}{%
		\providecommand*{\@course}{\VariableNotSetFix{\course}}%
		
	}{}
	

	\hypersetup{pageanchor=false}
	\begin{titlepage}%
		\vspace*{1cm}
		
		\begin{figure}[H]
			\centering
			\includegraphics[width=3cm]{weiti_logo.png} % Include a department/university logo - this will require the graphicx package
			\label{fig:logo}
		\end{figure}
		\begin{center}%
			\@ifundefined{@university}{}{%
				{\small \@university \par}}%
			\@ifundefined{@faculty}{}{%
				{\small \@faculty \par}}%
		\end{center}
		%\vspace{1cm}
		\begin{center}%
			\@ifundefined{@coursecode}{%
				{\Large \@course \par}%
			}{%
			{\Large \@course \ (\@coursecode) \par}}%
	\end{center}
	\sepline
	\begin{center}%
		{\LARGE \textsc{\@title} \par}
		\@ifundefined{@subtitle}{}%
		{\vspace{0.5cm}%
			\normalsize \textsc{\@subtitle} \par}%
	\end{center}
	\sepline
	\vspace{1cm}
	\begin{center}%
		{\normalsize
			\itshape
			\lineskip .5em%
			\begin{tabular}[t]{c}%
				\@author
			\end{tabular}\par}%
	\end{center}
	\vspace{2cm}
	\@ifundefined{@supervisor}{}{%	
		\begin{flushright}%
			{\normalsize Opiekun projektu: \par}
			{\normalsize \itshape \@supervisor \par}
		\end{flushright}		
		
	}
	%\vfill\null
	\begin{center}%
		\@date
	\end{center}
	
	\@thanks
\end{titlepage}%
\hypersetup{pageanchor=true}
\setcounter{footnote}{0}%
\global\let\thanks\relax
\global\let\maketitle\relax
\global\let\@thanks\@empty
\global\let\@author\@empty
\global\let\@date\@empty
\global\let\@title\@empty
\global\let\title\relax
\global\let\author\relax
\global\let\date\relax
\global\let\and\relax

}

\makeatother

\title{Teoria grafów w~modelowaniu epidemii}
\subtitle{Sprawozdanie 2}
\author{%
	Michał Dobrzański\\
	\texttt{\href{mailto:mdobrzan@mion.elka.pw.edu.pl}%
		{\nolinkurl{mdobrzan@mion.elka.pw.edu.pl}}}
	\and
	Maciej Janusz Krajsman\\
	\texttt{\href{mailto:M.Krajsman@stud.elka.pw.edu.pl}%
		{\nolinkurl{M.Krajsman@stud.elka.pw.edu.pl}}}
}
\supervisor{mgr inż. Łukasz Błaszczyk}
\university{Politechnika Warszawska}
\faculty{Wydział Elektroniki i~Technik Informacyjnych}
\course{Grafy i~Sieci}
\coursecode{GIS}

\begin{document}

\maketitle
\tableofcontents
\clearpage

\section{Szczegółowy opis merytoryczny zadania}
\label{sec:szczegolowy_opis_merytoryczny_zadania}

Zadaniem projektowym jest \textbf{zaimplementowanie modelu SIS} rozwoju epidemii, zbadanie jego właściwości na grafach modelujących populację ludzi, w~których wierzchołek odpowiada konkretnej osobie. Należy również porównać otrzymane wyniki z~modelem ciągłym dla epidemii opisanym równaniami różniczkowymi.  

Zadanie będzie polegało na \textbf{wygenerowaniu odpowiednich grafów losowych}, a~następnie na uruchomieniu \textbf{algorytmu propagacji epidemii} (zgodnego z~modelem SIS) dla wierzchołków tych grafów.

Otrzymane wyniki zostaną zestawione z~wynikami otrzymanymi za pomocą równań różniczkowych. Dla przejrzystości utworzone zostaną wykresy porównujące oba podejścia. Dobór prezentowanych parametrów dla osi wykresów zostanie określony w~trakcie tworzenia projektu.


\section{Opis algorytmu generowania grafów losowych według modelu sieci Barabásiego-Albert}
\label{sec:opis_alg_ba}

Do zrealizowania projektu potrzebny będzie algorytm generujący grafy losowe według \textbf{modelu sieci Barabásiego-Albert}. Utworzone sieci dzięki temu modelowi są nazywane sieciami przypadkowymi ewoluującymi.

Zaproponowana przez autorów \textbf{ewolucja sieci} rozumiana jako zmiany struktury sieci w~kolejnych odstępach czasowych modelowana jest poprzez dołączanie nowych węzłów do istniejącej już sieci. 

Procedura konstruowania sieci Barabásiego-Albert obejmuje następujące kroki:

\begin{enumerate}
\item Na początkowym etapie ewolucji (czyli w~chwili $t = 0$) siecią nazywamy graf pełny (całkowicie połączony klaster węzłów) o~rozmiarze: $m_0 >= 1$. W~następnych krokach czasowych $t = 1,2,3,...$ do sieci dodawane są nowe węzły (jeden węzeł na jeden krok), które tworzą odpowiednio $m <= m_0  (m = const)$ połączeń (czyli krawędzi) do istniejących już wierzchołków sieci.
\item Proces dodawania wierzchołków realizuje \textbf{regułę preferencyjnego dołączania}, która mówi o~tym, że prawdopodobieństwo, że nowy wierzchołek utworzy połączenie do jednego ze starszych wierzchołków jest wprost proporcjonalne do stopnia wierzchołka starszego.
\item Wzrost sieci kończony jest w~dowolnej chwili t. W~momencie zakończenia wzrostu sieć ma: \[N = t + m_0 \approx t \] wierzchołków (węzłów) oraz \[ E = mt + \binom{m_0}{2} \approx mt \] krawędzi.
\end{enumerate}


\section{Szczegółowy opis modelu SIS rozwoju epidemii}
\label{sec:szczegolowy_opis_modelu_sis_rozwoju_epidemii}

\textbf{Model SIS} opiera się na zestawie równań różniczkowych, które opisują \textbf{rozprzestrzenianie się chorób zakaźnych}. Służy do określania, czy dana choroba \textbf{zaniknie}, czy \textbf{ustali się} na konkretnym poziomie. Jest najbardziej ogólnym modelem matematycznym tego zjawiska --- nie uwzględnia np. przypadków nosicielstwa i~uodpornienia, istnienia osobników w~fazie utajonej choroby czy urodzeń chorych. Stan populacji w~tym modelu opisany jest \textbf{tylko jedną zmienną (\textit{I})}, która reprezentuje liczbę osobników zainfekowanych.

\subsection{Założenia modelu SIS}
\label{subsec:zalozenia_modelu_sis}

\begin{description} \itemsep0pt
\item[$\bullet$] Jednostki mogą znajdować się w~jednym z~dwóch stanów:
  \begin{description} \itemsep0pt
  \item[S] \textit{(susceptible --- ang. podatni)} --- jednostki są zdrowe, podatne na zakażenie.
  \item[I] \textit{(infected --- ang. zainfekowani)} --- jednostki są chore, mogą zakażać zdrowe jednostki.
  \end{description}
\item[S(t)] --- liczba osobników podatnych (zdrowych), tj. liczba węzłów w~stanie S w~danej chwili czasowej $t$.
\item[I(t)] --- liczba jednostek zainfekowanych, tj. liczba węzłów w~stanie I~w~danej chwili czasowej $t$.
\item[$\bullet$] Do zakażenia (zmiany stanu węzła z~$S$ na $I$ może dojść na skutek kontaktu jednostki zdrowej i~chorej.
\item[N] --- liczebność populacji, dla której $N = S(t) + I(t)$, Liczebność populacji jest stała, tj. $N = \textit{const.}$

\item[$\pmb{\beta}$] --- prawdopodobieństwo, że w~pojedynczym kroku czasowym ($dt$) zdrowy osobnik zakazi się od chorego sąsiada.
\item[$\pmb{\gamma}$] --- prawdopodobieństwo, że w~pojedynczym kroku czasowym ($dt$) chory osobnik wyzdrowieje, tj. zmieni stan $ I~\rightarrow S $.
\item[$\pmb{\lambda}$] --- parametr określający \textbf{tempo rozprzestrzeniania się epidemii}. Jest to stosunek $\lambda = \beta / \gamma $.
\item[$\pmb{\lambda_c}$] --- \textbf{wartość progowa tempa rozprzestrzeniania się epidemii}. Gdy tempo przekroczy tą wartość krytyczną, wówczas epidemia ma szansę stać się powszechną, a~badana choroba nabrać charakteru endemicznego. Dla symulacji w~czasie nieograniczonym oznacza to endemię.

\end{description}


\subsection{Warunki początkowe}
\label{subsec:warunki_poczatkowe}

Dla sieci losowych oraz bezskalowych (BA) dzięki modelowi SIS można opisać propagację epidemii za pomocą równania różniczkowego określającego tempo zmiany w~czasie liczby zakażonych węzłów o~zadanym stopniu k. W~tym celu należy określić \textbf{warunki początkowe} epidemii, czyli w~chwili $t = 0$. Określa się:
\begin{description} \itemsep0pt
\item[P(k)] --- rozkład stopni wierzchołków otrzymany po wygenerowaniu grafu BA dla wszystkich wartości $k$.
\item[$\pmb{\beta, \gamma}$] --- określone prawdopodobieństwa odpowiednio: zdrowy zostanie zakażony przez chorego, chory wyzdrowieje.
\item[$\pmb{I_k(0)}$] --- początkowa liczba zainfekowanych węzłów o~stopniu $k$.
\item[$\pmb{S_k(0)}$] --- początkowa liczba podatnych węzłów o~tym samym stopniu $k$.
\end{description}

Pytanie, jakie zadaje się przy modelowaniu epidemii najczęściej brzmi: Czy dla zadanych wartości parametrów $\beta$ i~$\lambda$ oraz dla zadanej początkowej liczby zainfekowanych osobników $i_0(0)$ infekcja rozprzestrzeni się czy nie?

Ponadto dla sieci bezskalowych szuka się wartości parametru $\lambda_c$, czyli progu od którego epidemia może stać się powszechna w~danej populacji.

\subsection{Równania różniczkowe opisujące model}
\label{subsec:rownania_rozniczkowe_opisujace_model}

Oprócz wcześniej wymienionych parametrów dla modelu SIS w~sieciach bezskalowych określa się prawdopodobieństwo, że dowolna krawędź grafu prowadzi do węzła, który przechowuje stan o~wartości ,,chory''.

\begin{equation}
\label{eq:q_i}
Q_I = \sum_{k} Q(k) i_k
\end{equation}
Gdzie:
\begin{description} \itemsep0pt
\item[k] --- stopień wierzchołka.
\item[$\pmb{Q_I}$] --- prawdopodobieństwo, że dowolna krawędź grafu prowadzi do węzła, który przechowuje stan o~wartości ,,chory''
\item[$\pmb{Q(k)}$] --- prawdopodobieństwo, że dowolna krawędź grafu prowadzi do węzła o~stopniu $k$, który przechowuje stan o~wartości ,,chory''.
\item[$\pmb{i_k}$] --- prawdopodobieństwo, że węzeł o~stopniu $k$ przechowuje stan o~wartości ,,chory''.
\end{description}

Wartość $Q(k)$ można dla danej sieci powiązań dla danej populacji wyliczyć z~następującego wzoru:
\begin{equation}
\label{eq:q_k}
Q(k) = \frac{k}{\langle k \rangle}P(k)
\end{equation}
Gdzie:
\begin{description} \itemsep0pt
\item[k] --- stopień wierzchołka.
\item[$\pmb{\langle k \rangle}$] --- gęstość sieci. Średni stopień losowo wybranego wierzchołka.
\end{description}

W przypadku sieci BA o~tworzeniu połączeń w~sieci nie decyduje żadne z~góry narzucone prawo, jej \textbf{struktura zależy od indywidualnych decyzji jej elementów} (kto kogo poznał, kto się z~kim widuje). W~rezultacie otrzymujemy \textbf{potęgowy rozkład stopni wierzchołków}, w~którym w~bliskim sąsiedztwie mogą występować wierzchołki o~ogromnej liczbie incydentnych krawędzi oraz te niemalże odizolowane od reszty sieci. Ze względu na potęgowy rozkład stopni wierzchołków, wyniki byłyby niewłaściwe. Równania korzystające ze średniego stopnia \emph{$\langle k \rangle$} ilustrują dynamikę układu zawierającego same wierzchołki o~średnim stopniu. W~opisie dynamiki sieci BA należy więc stosować równania dotyczące wierzchołków o~\textbf{zadanym stopniu \emph{k}}.

Tempo zmiany osobników zainfekowanych (węzłów o~zadanym stopniu $k$) $I_k(t)$ opisuje się następująco:

\begin{equation}
\label{eq:dii_k}
\frac{dI_k(t)}{dt} = [\beta k Q_I]S_k(t) - \gamma I_k(t)
\end{equation}

Można również wyliczać tempo zmiany prawdopodobieństwa, że konkretny osobnik (węzeł o~zadanym stopniu $k$) jest zainfekowany:
\begin{equation}
\label{eq:di_k}
\frac{di_k(t)}{dt} = [\beta k Q_I]s_k(t) - \gamma i_k(t);~~i_k(t) + s_k(t) = 1
\end{equation}
% Gdzie: $$
% \begin{itemize} \itemsep0pt
% \item 
% \end{itemize}

Szczególnie interesujący jest przypadek graniczny, dla $\displaystyle i_k =  \lim_{t\to\infty} i_k(t)$:
\begin{equation}
\label{eq:i_k_infty}
i_k = \frac{\lambda k Q_I}{1+\lambda k Q_I}
\end{equation}
Zgodnie z~zapisem w~sekcji \ref{subsec:zalozenia_modelu_sis}, parametr $\lambda = \beta / \gamma $ określa tempo rozprzestrzeniania się epidemii.
W powyższym równaniu (\ref{eq:i_k_infty}) określone zostało prawdopodobieństwo że węzeł o~stopniu \emph{k} jest w~stanie \emph{I} (jest zakażony) w~czasie nieskończonym, czyli po zakończeniu symulacji. Wyzerowanie $i_k$ zależy od $Q_I$, ponieważ \emph{k} jest dodatnią liczbą naturalną, a~$\lambda > 0$.
% Możemy dzięki temu stwierdzić, czy badana choroba utrzymuje się na stałym poziomie, czy zanika. 
By wykonywać obliczenia dla wszystkich stopni \emph{k}, należy zsumować obliczenia dla każdego z~nich. Z~wzorów (\ref{eq:q_i}), (\ref{eq:q_k}) i~ (\ref{eq:i_k_infty}) wynika, że:

\begin{equation}
\label{eq:q_i}
Q_I = \sum_{k}Q(k)i_k = 1 - \frac{1}{\langle k \rangle}\sum_{k}P(k)\frac{k}{1+\lambda k Q_I} = f(Q_I)
\end{equation}
Gdzie:
\begin{description} \itemsep0pt
\item[k] --- stopień wierzchołka.
\item[$\pmb{f(Q_I)}$] --- funkcja rozkładu prawdopodobieństwa przechowywania stanu ,,chory'' przez węzły o~różnych stopniach.
\end{description}
Powyższe równanie może mieć jedno lub dwa rozwiązania. Jednym jest zawsze $Q_I = 0$ (dla $i_k = 0$). Obecność drugiego rozwiązania świadczy o~tym, że losowo wybrana krawędź prowadzi do zainfekowanego wierzchołka. Epidemia może się swobodnie rozwijać, więc $\lambda > \lambda_c$. Dzięki temu, warunek:

\begin{equation}
\label{eq:q_i_mt_one}
\left.\frac{df(Q_I)}{dQ_I}\right|_{Q_I=0} > 1
\end{equation}

umożliwia wyznaczenie $\lambda_c$. By funkcja (\ref{eq:q_i}) spełniała warunek (\ref{eq:q_i_mt_one}):

\begin{equation}
\label{eq:q_i_mt_one_q_i}
\left.\frac{df(Q_I)}{dQ_I}\right|_{Q_I=0} = \left.\frac{1}{\langle k \rangle}\sum_{k}P(k)\frac{\lambda k^2}{(1+\lambda k Q_I)^2}\right|_{Q_I=0} = \frac{\lambda}{\langle k \rangle}\sum_{k}k^2P(k) = \lambda\frac{k^2}{k} > 1 
\end{equation}

Więc:
\begin{equation}
\label{eq:lambda_c}
\lambda_c = \frac{\langle k \rangle}{\langle k^2 \rangle}
\end{equation}


Okazuje się, że w~sieciach bezskalowych wartość parametru $\lambda_c$ jest bardzo zbliżona do zera. W~praktyce powinno to oznaczać, że powstrzymanie rozwoju epidemii jest prawie niewykonalne.


% \begin{equation}
% \label{eq:dSdt}
% \frac{dS}{dt} = -\frac{\beta SI}{N}+\mu(N-S)+\gamma I
% \end{equation}

% \begin{equation}
% \label{eq:dIdt}
% \frac{dI}{dt} = \frac{\beta SI}{N}-\gamma I-\mu I
% \end{equation}

% Gdzie ($\ref{eq:dSdt}$) określa tempo zmian liczebności grupy S (zdrowych), a~($\ref{eq:dIdt}$) --- w~grupie I~(chorych).


\section{Opis planu zastosowania grafów losowych do omawianego zagadnienia}
\label{sec:opis_planu_zastosowania_grafow_losowych}

Wygenerowane grafy losowe według modelu Barabasiego-Arbert posłużą za model badanej populacji. \textbf{Wierzchołkami} będziemy reprezentowali poszczególne osoby w~danej populacji. \textbf{Krawędzie} będą przedstawiały natomiast relacje pomiędzy tymi osobami. Określamy, że dwa wierzchołki są sąsiednie, gdy mają wspólną krawędź grafu.

\textbf{Siecią losową, bezskalową} (w zadanej chwili $t$ utworzoną według modelu BA) będziemy nazywać określoną populację o~ustalonej liczności, która zostanie poddana modelowaniu zjawiska propagacji epidemii. 

Wierzchołki grafu będą przechowywały informacje o~stanie osoby --- czy jest ona zdrowa (podatna), czy chora (zakażająca) według modelu SIS. Zostanie zastosowana reprezentacja grafu za pomocą listy wierzchołków wraz z~odpowiadającymi im listami sąsiedztwa. Badanie rozwoju epidemii będzie bazowało na tej reprezentacji opisującej stan, w~jakim się dany graf znajduje.

\section{Podstawowe założenia implementowanego programu oraz projekt testów}
\label{sec:podstawowe_zalozenia_programu}

Program zostanie zaimplementowany w~środowisku Python. Do jego realizacji użyje się bibliotek niezbędne do rysowania wykresów oraz grafów.

\textbf{Struktura grafu} będzie reprezentowana za pomocą listy sąsiedztwa. Dodatkowo w~niej dla każdego wierzchołka umieści się dodatkową informacją o~jego aktualnym stanie ($S$ lub $I$ według modelu SIS). W~środowisku Python będzie to słownik tworzony za pomocą konstruktora \textit{dict()}. Badanie algorytmu propagacji epidemii sprowadzi się do przechodzenia po listach sąsiedztwa i~zmianach stanów dla wierzchołków.

W implementowanym programie przeprowadzone zostaną następujące testy:
\begin{enumerate}
\item Sprawdzenie poprawności algorytmu generującego grafy losowe według modelu sieci Barabasiego-Albert:

Zostaną wygenerowane sieci losowe BA (o różnych parametrach $m_0$, $m$ oraz $t$. Ich rozkłady stopni wierzchołków zostaną przedstawione na wykresie. Prawidłowo utworzone sieci według modelu BA powinny charakteryzować się w~przybliżeniu \textbf{potęgowym rozkładem stopnia wierzchołka}. Ten rozkład przybiera postać liniowego przebiegu na wykresie w~skali podwójnie logarytmicznej. Na tym wykresie wartość nachylenia krzywej przekłada się na wartość wykładnika dla rozkładu potęgowego. W~programie utworzony zostanie taki wykres i~przedstawione na nim będzie kilka przykładowych rozkładów dla różnych sieci losowych BA.

\item Przeprowadzenie symulacji rozwoju epidemii według modelu SIS dla różnych parametrów:

Następnie dla wcześniej wygenerowanych sieci losowych przeprowadzona zostanie symulacja rozwoju epidemii. Dla każdej sieci utworzone będzie kilka różnych warunków początkowych (odpowiednia liczba wierzchołków w~stanie $S$ i~$I$ w~chwili $t_0$, różne parametry $\gamma$, $\beta$). Następnie dla kolejnych chwil $t = 1,2,3,...$ będzie badany rozwój epidemii. Najistotniejszy jest wynik dla $t\to\infty$, ponieważ opisuje stan końcowy układu. Wyniki porównane zostaną ze spodziewanym rezultatem otrzymanym z~równań opisujących model SIS. Sprawdzone zostanie również, jak model zachowuje się dla $\lambda > \lambda_c$ oraz $\lambda < \lambda_c$ (w tym celu graf nie może być za mały, a~wartości $\lambda_c$ i~tak będą bardzo niewielkie, ze względu na potęgowy rozkład stopni wierzchołków grafu). 


\end{enumerate}

\clearpage
%*******************************************************************************
% Bibliografia - spis literatury wykorzystanej przy tworzeniu pracy
%*******************************************************************************
%\clearpage
\begin{thebibliography}{99}
\addcontentsline{toc}{chapter}{Bibliografia}

%1
\bibitem{SieciZlozone} Fronczak Agata, Fronczak Piotr: \emph{,,Świat sieci złożonych''}, 
\newline Wyd. I, Warszawa, wyd. PWN, 2009, ISBN: 978-83-01-15987-0.

%2
\bibitem{EpiWSieciach} Fronczak Agata, Fronczak Piotr: \emph{,,Epidemie w sieciach złożonych''} [online],  \\ 28.10.2010,
\url{ http://www.if.pw.edu.pl/~agatka/moodle/epidemie.html } , \\ dostęp: 20.11.2016r.



\end{thebibliography}

\clearpage

%===============================================================================


\end{document}